\documentclass{article}
\usepackage[utf8]{inputenc}

\title{Project Topic - Introducing the State of Research on the Benguela Upwelling System}
\author{Nathan Knauf, Daniel Muratore}
\date{March 2021}

\begin{document}

\maketitle

\section{Concept}
\paragraph{Introduction} Microorganisms in the ocean conduct approximately half of all photosynthesis that takes place on Earth, having a major impact on the global carbon cycle and the Earth system's response to anthropogenic carbon dioxide emissions. Those microorganisms (phytoplankton) also form the base of the marine food web, supporting the global fisheries industry, which is estimated to employ approximately 60 million people globally, largely concentrated in postcolonial nations in Africa and Asia, and generating on the order of 1.5 trillion dollars of value annually. Critically, the spatial arrangement of the ocean's biological production is not at all uniform. The necessary nutrients for phytoplankton growth quickly deplete at the ocean surface and tend to collect in the deep ocean. Global wind patterns and ocean circulation push nutrient-rich deep water to the surface primarily in four locations around the world, termed 'Eastern Boundary Upwelling Systems'. These four locations (off the coasts of California, Peru/Chile, the Canary Islands and Iberian Peninsula, and Angola/Namibia/South Africa) account for only 3\% of ocean surface area but produce almost 40\% of the global fish catch. Our project aims to introduce 
\subsection{What Problem is Being Addressed?}
\subsection{Who is interested in this?}
\subsection{What do people want to know about these data?}
\section{Data}
\paragraph{}
\subsection{Data Sources}
\paragraph{Oceanographic Data}
Contextual oceanographic data for geospatial visualizations are collected from MODIS satellite observations stored at the NASA earthdata portal. We collected imputed chlorophyll a concentrations (chlorophyll being a standard proxy for biological activity as it is proportional to the biomass of phytoplankton, which form the base of the marine foodchain) as well as surface wind and velocity data (to show the direction of currents and upwelling strength) for the year of 2020. These data are stored as netcdf files but can handily be converted to long tables of lat, long, time, (variables). A snippet for chlorophyll data is shown below. These data are in a format readable to tableau, which is friendly to making ocean maps. 
\paragraph{Scientific Publication Data}

\paragraph{Economic Data}
\subsection{Data Snippets}

\end{document}
